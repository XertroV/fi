% ---
% title: "Reply to 'Mathematical Inconsistency in Solomonoff Induction?'"
% date: 2020-09-04
% parent: Posts
% layout: post
% published: true
% ---


\documentclass{article}
\usepackage{footnote}
% \usepackage[utf8]{inputenc}
\usepackage{hyperref}
\usepackage{amsmath}
\usepackage{amssymb}
\usepackage{cancel}
\usepackage[capitalise,nameinlink,noabbrev]{cleveref}
\usepackage{enumitem}
\setitemize{noitemsep,topsep=0pt,parsep=0pt,partopsep=0pt}

\begin{document}

This is a reply to \href{https://www.lesswrong.com/posts/hD4boFF6K782grtqX/mathematical-inconsistency-in-solomonoff-induction}{this LessWrong post}.

I went through the maths in OP and it seems to check out. I think the core inconsistency is that SI implies \(l(X \cup Y) = l(X)\). I'm going to redo the maths below (breaking it down step-by-step more). curi has \(2l(X) = l(X)\) which is the same inconsistency given his substitution. I'm not sure we can make that substitution but I also don't think we need to.

Let $X$ and $Y$ be independent hypotheses for Solomonoff induction.

According to the prior, the non-normalized probability of $X$ (and similarly for $Y$) is:

% \newcommand{\SiP}[1]{
%     \frac{1}{2^{l(#1)}}
% }

\begin{equation} \label{eq_prior}
P(X) = \frac{1}{2^{l(X)}}
\end{equation}

what is the probability of \(X\cup Y\)?

\begin{equation} \label{eq_std_prob}
\begin{split}
P(X\cup Y) & = P(X) + P(Y) - P(X\cap Y) \\
& = \frac{1}{2^{l(X)}} + \frac{1}{2^{l(Y)}} - \frac{1}{2^{l(X)}} \cdot \frac{1}{2^{l(Y)}} \\
& = \frac{1}{2^{l(X)}} + \frac{1}{2^{l(Y)}} - \frac{1}{2^{l(X)} \cdot 2^{l(Y)}} \\
& = \frac{1}{2^{l(X)}} + \frac{1}{2^{l(Y)}} - \frac{1}{2^{l(X) + l(Y)}}
\end{split}
\end{equation}

However, by \cref{eq_prior} we have:

\begin{equation} \label{eq_or}
P(X\cup Y) = \frac{1}{2^{l(X\cup Y)}}
\end{equation}

thus

\begin{equation} \label{eq_both}
\frac{1}{2^{l(X\cup Y)}} = \frac{1}{2^{l(X)}} + \frac{1}{2^{l(Y)}} - \frac{1}{2^{l(X) + l(Y)}}
\end{equation}

This must hold for \emph{any and all} $X$ and $Y$.

curi considers the case where $X$ and $Y$ are the same length, starting with \cref{eq_both}

\begin{equation}
    \begin{split}
        \frac{1}{2^{l(X\cup Y)}} & = \frac{1}{2^{l(X)}} + \frac{1}{2^{l(Y)}} - \frac{1}{2^{l(X) + l(Y)}} \\
        & = \frac{1}{2^{l(X)}} + \frac{1}{2^{l(X)}} - \frac{1}{2^{l(X) + l(X)}} \\
        & = \frac{2}{2^{l(X)}} - \frac{1}{2^{2l(X)}} \\
        & = \frac{1}{2^{l(X)-1}} - \frac{1}{2^{2l(X)}}
    \end{split}
\end{equation}

but

\begin{equation}
\frac{1}{2^{l(X)-1}} \gg \frac{1}{2^{2l(X)}}
\end{equation}

and

\begin{equation}
0 \approx \frac{1}{2^{2l(X)}}
\end{equation}

so

\begin{equation} \label{eq_cont_1}
\begin{split}
\frac{1}{2^{l(X\cup Y)}} & \simeq \frac{1}{2^{l(X)-1}} \\
\therefore l(X\cup Y) & \simeq l(X)-1 \\
& \square
\end{split}
\end{equation}

curi has slightly different logic and argues \(l(X\cup Y) \simeq 2l(X)\) which I think is reasonable. His argument means we get \(l(X) \simeq 2l(X)\). I don't think those steps are necessary but they are worth mentioning as a difference. I think \cref{eq_cont_1} is enough.

I was curious about what happens when \(l(X) \neq l(Y)\). Let's assume the following:

\begin{equation} \label{eq_len_ineq}
\begin{split}
l(X) & < l(Y) \\
\therefore \frac{1}{2^{l(X)}} & \gg \frac{1}{2^{l(Y)}}
\end{split}
\end{equation}

so, from \cref{eq_std_prob}

\begin{equation} \label{eq_or_ineq}
    \begin{split}
P(X\cup Y) & = \frac{1}{2^{l(X)}} + \frac{1}{2^{l(Y)}} - \frac{1}{2^{l(X) + l(Y)}} \\
\lim_{l(Y) \to \infty} P(X\cup Y) & = \frac{1}{2^{l(X)}} + \cancelto{0}{\frac{1}{2^{l(Y)}}} - \cancelto{0}{\frac{1}{2^{l(X) + l(Y)}}} \\
\therefore P(X\cup Y) & \simeq \frac{1}{2^{l(X)}}
    \end{split}
\end{equation}

by \cref{eq_or} and \cref{eq_or_ineq}

\begin{equation} \label{eq_cont_2}
    \begin{split}
\frac{1}{2^{l(X\cup Y)}} & \simeq \frac{1}{2^{l(X)}} \\
\therefore l(X\cup Y) & \simeq l(X) \\
\Rightarrow l(Y) & \simeq 0
    \end{split}
\end{equation}

but \cref{eq_len_ineq} says \(l(X) < l(Y)\) -- this contradicts \cref{eq_cont_2}. \(\blacksquare\)

So there's an inconsistency regardless of whether \(l(X) = l(Y)\) or not.

\end{document}
